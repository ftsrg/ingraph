\newcommand{\relnull}{\mathsf{NULL}}

\newcommand{\assign}{\rightarrow}

\newcommand{\lxor}{\oplus}

\newcommand{\asc}{\uparrow}
\newcommand{\desc}{\downarrow}

\newcommand{\tuple}[1]{\langle #1 \rangle}
\newcommand{\concatenation}{\Vert}

\newcommand{\literal}[1]{\mathsf{#1}}
\newcommand{\atom}[1]{\mathsf{#1}}

\newcommand{\var}[1]{\mathtt{#1}}
\newcommand{\edgevariable}[2]{\var{#1}
	\ifstrempty{#2}{}{\colon{\atom{#2}}}}
\newcommand{\nodevariable}[2]{(\var{#1}
	\ifstrempty{#2}{}{\colon{\atom{#2}}})}

% see http://tug.ctan.org/info/symbols/comprehensive/symbols-a4.pdf

%%%%%%%%%%%%%%%%%%%%% operator symbols %%%%%%%%%%%%%%%%%%%%%

\iftoggle{textualoperators}{
	% nullary operators
	\newcommand{\getverticesop}{\bigcirc}
	\newcommand{\getedgesopdirected}{\Uparrow}
	\newcommand{\getedgesopundirected}{\Updownarrow}
	
	% unary operators
	\newcommand{\expandbothop}{\updownarrow}
	\newcommand{\expandoutop}{\uparrow}
	\newcommand{\expandinop}{\downarrow}
	\newcommand{\transitiveclosurebothop}{\oplus \updownarrow}
	\newcommand{\transitiveclosureoutop}{\oplus \updownarrow}
	\newcommand{\transitiveclosureinop}{\oplus \updownarrow}
	\newcommand{\alldifferentop}{\not\equiv}
	\newcommand{\duplicateeliminationop}{\delta}
	\newcommand{\sortop}{\tau}
	\newcommand{\projectionop}{\pi}
	\newcommand{\selectionop}{\sigma}
	\newcommand{\renameop}{\rho}
	\newcommand{\groupingop}{\gamma}
	\newcommand{\topop}{\lambda}
	\newcommand{\unwindop}{\omega}
	
	% binary operators
	\newcommand{\joinop}{\bowtie}
	\def\ojoin{\setbox0=\hbox{$\bowtie$}%
		\rule[-.02ex]{.25em}{.4pt}\llap{\rule[\ht0]{.25em}{.4pt}}}
	\def\leftouterjoin{\mathbin{\ojoin\mkern-5.8mu\bowtie}}
	\def\rightouterjoin{\mathbin{\bowtie\mkern-5.8mu\ojoin}}
	\def\fullouterjoin{\mathbin{\ojoin\mkern-5.8mu\bowtie\mkern-5.8mu\ojoin}}
	
	\newcommand{\leftouterjoinop}{\leftouterjoin}
	\newcommand{\antijoinop}{\, \triangleright \,}
	\newcommand{\unionop}{\cup}
	\newcommand{\bagunionop}{\uplus}
	\newcommand{\minusop}{\setminus}
	\newcommand{\intersectionop}{\cap}
	\newcommand{\cartesianproductop}{\times}
}{
	\newcommand{\op}[1]{\textsc{#1}}
	
	% nullary operators
	\newcommand{\getverticesop}{\relalgop{GetVertices}}
	\newcommand{\getedgesopdirected}{\relalgop{GetEdges}}
	\newcommand{\getedgesopundirected}{\relalgop{GetEdgesUndirected}}
	
	% unary operators
	\newcommand{\expandbothop}{\relalgop{ExpandBoth}}
	\newcommand{\expandoutop}{\relalgop{ExpandOut}}
	\newcommand{\expandinop}{\relalgop{ExpandIn}}
	\newcommand{\transitiveclosurebothop}{\relalgop{TransitiveClosureBoth}}
	\newcommand{\transitiveclosureoutop}{\relalgop{TransitiveClosureOut}}
	\newcommand{\transitiveclosureinop}{\relalgop{TransitiveClosureIn}}
	\newcommand{\alldifferentop}{\relalgop{AllDifferent}}
	\newcommand{\duplicateeliminationop}{\relalgop{DuplicateElimination}}
	\newcommand{\sortop}{\relalgop{Sort}}
	\newcommand{\projectionop}{\relalgop{Projection}}
	\newcommand{\selectionop}{\relalgop{Selection}}
	\newcommand{\renameop}{\relalgop{Rename}}
	\newcommand{\groupingop}{\relalgop{Grouping}}
	\newcommand{\topop}{\relalgop{Top}}
	\newcommand{\unwindop}{\relalgop{Unwind}}
	
	% binary operators
	\newcommand{\joinop}{\relalgop{Join}}
	\newcommand{\leftouterjoinop}{\relalgop{LeftOuterJoin}}
	\newcommand{\antijoinop}{\relalgop{AntiJoin}}
	\newcommand{\unionop}{\relalgop{Union}}
	\newcommand{\bagunionop}{\relalgop{BagUnion}}
	\newcommand{\minusop}{\relalgop{Minus}}
	\newcommand{\intersectionop}{\relalgop{Intersection}}
	\newcommand{\cartesianproductop}{\relalgop{CartesianProduct}}
	
	% ternary operators
	%\newcommand{}{}
}

%%%%%%%%%%%%%%%%%%%%% operator definitions %%%%%%%%%%%%%%%%%%%%%

%%%%%%%%%% nullary operators %%%%%%%%%%

\newcommand{\getvertices}[2]{\getverticesop_{\nodevariable{#1}{#2}}}

\newcommand{\getedgesdirected}[6]{\getedgesopdirected_{\nodevariable{#1}{#2}}^{\nodevariable{#3}{#4}} \left[ \edgevariable{#5}{#6} \right]}
\newcommand{\getedgesundirected}[6]{\getedgesopundirected_{\nodevariable{#1}{#2}}^{\nodevariable{#3}{#4}} \left[ \edgevariable{#5}{#6} \right]}

%%%%%%%%%% unary operators %%%%%%%%%%
\newcommand{\nagivationbody}[3]{~_{\nodevariable{#1}{}}^{\nodevariable{#2}{#3}}}

\newcommand{\kleenestar}{\ast}

% expand operators
\newcommand{\expandedgevariable}[4]{
	\left[
	  % #3: minHops, cannot be empty
	  % #4: maxHops, if empty, default to infinity
	  \edgevariable{#1}{#2}
	  \ifstrequal{#3}{1} % minHops = 1
	  {
	  	\ifstrequal{#4}{1}
	  	{} % minHops = 1 and maxHops = 1 -> write nothing
	    {\kleenestar \atom{#3} \ldots \atom{#4}} % minHops = 1 and maxHops != 1
	  } % minHops != 1
      {\kleenestar \atom{#3} \ldots \atom{#4}}
	\right]}

\newcommand{\expandboth}[7]{\expandbothop \nagivationbody{#1}{#2}{#3} \expandedgevariable{#4}{#5}{#6}{#7} }
\newcommand{\expandout}[7]{\expandoutop \nagivationbody{#1}{#2}{#3} \expandedgevariable{#4}{#5}{#6}{#7} }
\newcommand{\expandin}[7]{\expandinop \nagivationbody{#1}{#2}{#3} \expandedgevariable{#4}{#5}{#6}{#7} }

% transitive closure operators
\newcommand{\transitiveclosureboth}[7]{\transitiveclosurebothop \nagivationbody{#1}{#2}{#3} \expandedgevariable{#4}{#5}{#6}{#7} }

\newcommand{\transitiveclosureout}[7]{\transitiveclosureoutop \nagivationbody{#1}{#2}{#3} \expandedgevariable{#4}{#5}{#6}{#7} }

\newcommand{\transitiveclosurein}[7]{\transitiveclosureinop \nagivationbody{#1}{#2}{#3} \expandedgevariable{#4}{#5}{#6}{#7} }

\newcommand{\alldifferent}[1]{\alldifferentop_{\atom{#1}}}
\newcommand{\duplicateelimination}{\duplicateeliminationop}
\newcommand{\sort}[1]{\sortop_{#1}}
\newcommand{\projection}[1]{\projectionop_\atom{#1}}
\newcommand{\selection}[1]{\selectionop_\atom{#1}}
\newcommand{\rename}[1]{\renameop_\atom{#1}}
\newcommand{\grouping}[1]{\groupingop_\atom{#1}}

% top/skip/limit operators
\newcommand{\topp}[2]{\topop_{#1}^{#2}}
\newcommand{\skipp}[1]{\topop^{#1}}
\newcommand{\limit}[1]{\topop_{#1}}

% see A Formal Presentation of MongoDB (Extended Version)
% by Elena Botoeva, Diego Calvanese, Benjamin Cogrel, Martin Rezk, Guohui Xiao
% https://arxiv.org/abs/1603.09291
\newcommand{\unwind}[2]{\unwindop_\atom{#1}^\atom{#2}}

%%%%%%%%%% binary operators %%%%%%%%%%

\newcommand{\join}{\joinop}
\newcommand{\myleftouterjoin}{\leftouterjoinop}
\newcommand{\antijoin}{\antijoinop}
\newcommand{\union}{\unionop}
\newcommand{\bagunion}{\bagunionop}
\newcommand{\minus}{\minusop}
\newcommand{\intersection}{\intersectionop}
\newcommand{\cartesianproduct}{\cartesianproduct}

% ternary operators
