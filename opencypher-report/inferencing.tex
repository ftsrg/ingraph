\chapter{Inferencing Tuples}

Graph query engines often use single-machine (even single-threaded), search-based algorithms. Using these algorithms includes many tradeoffs:

\begin{itemize}
	\item They do not scale well for large graphs.
	\item They cannot run in parallel.
	\item They cannot provide incremental maintenance -- if the graph changes, they have to completely reevaluate the query.
\end{itemize}

However, they have advantageous properties as well:

\begin{itemize}
	\item They are more straightforward to implement than incremental algorithms.
	\item They are typically able to execute with limited memory.
	\item They can access properties of elements (e.g. the \texttt{name} property of a vertex or the \texttt{weight} property of an edge) by using a pointer.
\end{itemize}

\subsection{Concepts}

Let's take a simple Cypher query, that returns the name :

\begin{lstlisting}[label=lst:inferencing-example, caption=Example query]
MATCH (n:Person)
WHERE n.age > 27
RETURN n.name
\end{lstlisting}

\newcommand{\screenshotscale}{0.45}

\begin{figure}
	\centering
	\includegraphics[scale=\screenshotscale]{neo4j-query-plan}
	\caption{Neo4j query plan for \autoref{lst:inferencing-example}.}
	\label{fig:neo4-query-plan}
\end{figure}

\begin{figure}
	\centering
	\input{example-inferencing}
	\caption{Relational algebra tree with tuples for \textcolor{gray}{external schema}, \textcolor{violet}{additional attributes} and \textcolor{orange}{internal schema}.}
	\label{fig:example-inferencing}
\end{figure}

There are three key concepts for defining the schema for each operator in the relational algebra plan:

\begin{itemize}
	\item The \emph{external schema} (typeset in \textcolor{gray}{gray}) is a relational schema that defines the schema visible for users. This schema is what the Neo4j query planner generates (see \autoref{fig:example-inferencing}).
	
	\item The \emph{additional attributes} (typeset in \textcolor{violet}{violet}) define attributes that are required by an operator's ancestors. For example, the \projectiontext ($\projectionop$) and \selectiontext ($\selectionop$) operators might require additional properties. For \autoref{lst:inferencing-example}, the projection requires the $\var{name}$ attribute, while the selection requires the $\var{age}$.
	
	\item The \emph{internal schema} (typeset in \textcolor{orange}{orange}) defines a relational schema that contains the \emph{external schema} plus the additional attributes that are required to locally perform computations.
\end{itemize}

\autoref{fig:neo4-query-plan} illustrates these concepts.

We call the process of calculating the external schema, the additional attributes and the internal schema as \emph{inferencing}.

\subsection{More examples}

See the Appendices for more examples, starting with the TCK test cases (\autoref{chp:tck}).
