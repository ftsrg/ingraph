% !TeX spellcheck = en_US
% !TeX program = xelatex

%Ensure that all odl school LaTeX habits are remarked
\RequirePackage[l2tabu, orthodox]{nag}
%Neue deutsche Trennmuster
%Siehe http://www.ctan.org/pkg/dehyph-exptl und http://projekte.dante.de/Trennmuster/WebHome
%Nur für pdflatex, nicht für lualatex
%\RequirePackage[ngerman=ngerman-x-latest]{hyphsubst}
%\documentclass{lni}
% in Englisch stattdessen:
\documentclass[english]{lni}

%% Some packages, no need to be adabted

% enable copy and paste of ligatures (e.g., in "workflow" and umlauts)
\usepackage{cmap}

%Überschrift des Literaturverzeichnisses
%Only in German
\iflnienglish
\else
\renewcommand{\refname}{Literaturverzeichnis}
\fi

%Enable input of umlauts using UTF-8.
\usepackage[utf8]{inputenc}

\usepackage{graphicx}
\usepackage{fancyhdr}

%Kopf- und Fußzeileneinstellungen
\fancypagestyle{lnifirstpage}{
% Löscht alle Kopf- und Fußzeileneinstellungen
\fancyhf{}

%Kopfzeile
\fancyhead[RO]{\small Submitted to: BTW 2017%,\linebreak%
}
%Für den Herausgeber:
%\fancyhead[RO]{\small <Vorname Nachname [et. al.]> (Hrsg.): <Buchtitel>,\linebreak%
%Lecture Notes in Informatics (LNI), Gesellschaft für Informatik, Bonn <Jahr> \hspace{5pt} \thepage \hspace{0.05cm}}

%Linie unter Kopfzeile
\renewcommand{\headrulewidth}{0.4pt}
}

% Put in the short title (Kurztitle) here
\fancypagestyle{lni}{
\fancyhf{}
%Zu lange Titel müssen von den HerausgeberInnen gekürzt werden, Vorschläge der AutorInnen dazu sind herzlich willkommen.
\fancyhead[RO]{\small \shorttitle \hspace{5pt}
\thepage \hspace{0.05cm}}
\fancyhead[LE]{\hspace{0.05cm}\small \thepage \hspace{5pt}
%Bis zu drei AutorInnen werden alle angeführt, darüber hinaus wird nur die erste Autorin bzw. der erste Autor angeführt und alle Weiteren mit et al.\ abgekürzt.
%Zu lange AutorInnenlisten müssen von den HerausgeberInnen gekürzt werden.
Gábor Szárnyas and József Marton}
\renewcommand{\headrulewidth}{0.4pt}
}

\iflnienglish
\usepackage[figurename=Fig., tablename=Tab., small]{caption}
\else
\usepackage[figurename=Abb., tablename=Tab., small]{caption}
\fi

\usepackage{xcolor}

%if lstlistings is used
%better approach: use the minted package - see https://en.wikibooks.org/wiki/LaTeX/Source_Code_Listings#The_minted_package
%SzG: updating minted is quite cumbersome, see http://tex.stackexchange.com/questions/18083/how-to-add-custom-c-keywords-to-be-recognized-by-minted
\usepackage{listings}

\definecolor{lightgray}{RGB}{242,242,242}
\definecolor{keywordcolor}{RGB}{0,0,160}
\definecolor{commentcolor}{RGB}{0,128,64}
\definecolor{stringcolor}{RGB}{0,128,0}
\lstset{
	numbers=left,
	numberstyle=\scriptsize\ttfamily,
	stepnumber=1,
	numbersep=5pt,
	%
	backgroundcolor=\color{lightgray},
	basicstyle=\scriptsize\ttfamily, % print whole listing small
	keywordstyle=\color{keywordcolor}\bfseries\ttfamily,
	commentstyle=\color{commentcolor}\ttfamily,
	stringstyle=\color{stringcolor}\ttfamily,
	identifierstyle=, % nothing happens
	%
	showstringspaces=false, % no special string spaces
	aboveskip=3pt,
	belowskip=3pt,
	columns=flexible,
	keepspaces=true,
	breaklines=true,	
	frameround=tttt,
	captionpos=b,
	tabsize=2,
	frame=tb,
	framerule=0pt,
	framexleftmargin=0.25em,
}

\lstdefinelanguage{cypher}
{
	morekeywords={
		MATCH, OPTIONAL, WHERE, NOT, AND, OR, XOR, RETURN, DISTINCT, ORDER, BY, ASC, ASCENDING, DESC, DESCENDING, UNWIND, AS, UNION, WITH, ALL, CREATE, DELETE, DETACH, REMOVE, SET, MERGE, SET, SKIP, LIMIT,
		% some legacy rules
		INDEX, DROP, UNIQUE, CONSTRAINT, EXPLAIN, PROFILE, START, CASE,
		% some SQL-only keywords
		GROUP, HAVING,
	},
	sensitive=true,
	morecomment=[l]{//},
	morecomment=[s]{/*}{*/},
	morestring=[b]{"},
}

\newcommand{\listingcypher}[2]{
	\lstset{
		language=Cypher
	}
	\lstinputlisting[label=lst:#1, caption=#2.]{queries/#1.cyp}
}

\definecolor{relationship}{RGB}{171,0,85}
\lstset{
	basicstyle=\small\ttfamily,
	tabsize=4,
	morekeywords={Class, encapsulates, Feature},
%	literate=*
%	{[:gathers]}{\textcolor{relationship}{[:gathers]}}{1}
}


\lstset{language=Cypher,
	literate=*
	{<v>}{\guillemotleft{}}{1}
	{</v>}{\guillemotright{}}{1},
%	{<v>}{\guilsinglleft{}}{1}
%	{</v>}{\guilsinglright{}}{1},
}

% Listingname heißt nun List.
\renewcommand{\lstlistingname}{List.}

%for easy quotations: \enquote{text}
\usepackage{csquotes}

\usepackage[T1]{fontenc}

%enable margin kerning
\usepackage{microtype}

%for demonstration purposes only
\usepackage[math]{blindtext}

%tweak \url{...}
\usepackage{url}
%nicer // - solution by http://tex.stackexchange.com/a/98470/9075
\makeatletter
\def\Url@twoslashes{\mathchar`\/\@ifnextchar/{\kern-.2em}{}}
\g@addto@macro\UrlSpecials{\do\/{\Url@twoslashes}}
\makeatother
\urlstyle{same}
%improve wrapping of URLs - hint by http://tex.stackexchange.com/a/10419/9075
\makeatletter
\g@addto@macro{\UrlBreaks}{\UrlOrds}
\makeatother

%diagonal lines in a table - http://tex.stackexchange.com/questions/17745/diagonal-lines-in-table-cell
%slashbox is not available in texlive (due to licensing) and also gives bad results. Thus, we use diagbox
%\usepackage{diagbox}

\usepackage{booktabs}
\usepackage{multirow}
\usepackage{array}
\usepackage{rotate}
\usepackage{enumitem}
\usepackage{changes}
\setlist{
	itemsep=-0.25em,
	topsep=-0.3em}

% new packages BEFORE hyperref
% See also http://tex.stackexchange.com/questions/1863/which-packages-should-be-loaded-after-hyperref-instead-of-before

%enable hyperref without colors and without bookmarks
\usepackage[
%pdfauthor={},
%pdfsubject={},
%pdftitle={},
%pdfkeywords={},
bookmarks=false,
breaklinks=true,
%colorlinks=true,
%linkcolor=black,
%citecolor=black,
%urlcolor=black,
pdfpagelayout=SinglePage,
pdfstartview=Fit
]{hyperref}
%enables correct jumping to figures when referencing
\usepackage[all]{hypcap}

%enable nice comments
\usepackage{pdfcomment}
\newcommand{\commentontext}[2]{\colorbox{yellow!60}{#1}\pdfcomment[color={0.234 0.867 0.211},hoffset=-6pt,voffset=10pt,opacity=0.5]{#2}}
\newcommand{\commentatside}[1]{\pdfcomment[color={0.045 0.278 0.643},icon=Note]{#1}}

%compatibality with TODO package
\newcommand{\todo}[1]{\commentatside{#1}}

%enable \cref{...} and \Cref{...} instead of \ref: Type of reference included in the link
\iflnienglish
\usepackage[capitalise,nameinlink]{cleveref}
%Nice formats for \cref
\crefname{section}{Sect.}{Sect.}
\Crefname{section}{Section}{Sections}
\crefname{figure}{Fig.}{Fig.}
\Crefname{figure}{Figure}{Figures}
\else
\usepackage[capitalise,nameinlink]{cleveref}
\fi

%introduce \powerset - hint by http://matheplanet.com/matheplanet/nuke/html/viewtopic.php?topic=136492&post_id=997377
%\DeclareFontFamily{U}{MnSymbolC}{}
%\DeclareSymbolFont{MnSyC}{U}{MnSymbolC}{m}{n}
%\DeclareFontShape{U}{MnSymbolC}{m}{n}{
%    <-6>  MnSymbolC5
%   <6-7>  MnSymbolC6
%   <7-8>  MnSymbolC7
%   <8-9>  MnSymbolC8
%   <9-10> MnSymbolC9
%  <10-12> MnSymbolC10
%  <12->   MnSymbolC12%
%}{}
%\DeclareMathSymbol{\powerset}{\mathord}{MnSyC}{180}

%improve float placement
%source: http://people.cs.uu.nl/piet/floats/node1.html
%see also: http://tex.stackexchange.com/a/35130/9075
\renewcommand{\textfraction}{0.05}
\renewcommand{\topfraction}{0.95}
\renewcommand{\bottomfraction}{0.95}
\renewcommand{\floatpagefraction}{0.35}
\setcounter{totalnumber}{5}

% correct bad hyphenation here
\hyphenation{net-works semi-conduc-tor}

%%% Adapt to your needs from here

%Beginn der Seitenzählung für diesen Beitrag
\setcounter{page}{1}


\newcommand{\retescale}{0.45}
\newcommand{\patternscale}{0.28}
\newcommand{\diagramscale}{0.75}
\newcommand{\phasesscale}{0.5}

\newcommand{\mysf}[1]{$\mathsf{#1}$}
\newcommand{\query}[1]{\mysf{#1}\xspace}

\newcommand{\connectedsegments}{\query{ConnectedSegments}}
\newcommand{\poslength}{\query{PosLength}}
\newcommand{\routesensor}{\query{RouteSensor}}
\newcommand{\switchmonitored}{\query{SwitchMonitored}}
\newcommand{\switchset}{\query{SwitchSet}}
\newcommand{\semaphoreneighbor}{\query{SemaphoreNeighbor}}

\newcommand{\iq}{\mbox{\textsc{IncQuery}}\xspace}
\newcommand{\iqd}{\mbox{\textsc{IncQuery-D}}\xspace}
\newcommand{\eiq}{\mbox{\textsc{EMF-IncQuery}}\xspace}
\newcommand{\viatraquery}{\mbox{\textsc{Viatra} Query}\xspace}
\newcommand{\vql}{\mbox{\textsc{Viatra} Query Language}\xspace}
\newcommand{\tb}{Train Benchmark\xspace}
\newcommand{\antlr}{\mbox{ANTLR4}\xspace}
\newcommand{\saphana}{SAP \mbox{HANA}\xspace}
\newcommand{\ingraph}{\textsf{ingraph}\xspace}
\newcommand{\opencypher}{\mbox{openCypher}\xspace}
\newcommand{\cypher}{\mbox{Cypher}\xspace}
\newcommand{\sparql}{\mbox{SPARQL}\xspace}
\newcommand{\rga}{relational graph algebra\xspace}
\newcommand{\RGA}{Relational Graph Algebra\xspace}

\newcommand{\queryplanscale}{0.9}

\newcommand{\fig}[4]{
	\centerline{\includegraphics[scale=#4]{#1}}
	\captionof{figure}{#3.}\label{fig:#2}
}

\newcommand{\queryplan}[2]{\fig{../ingraph/visualization/#1.pdf}{#1}{#2}{\queryplanscale}}

\newcommand{\eg}{e.g.\xspace}
\newcommand{\ie}{i.e.\xspace}
\newcommand{\etal}{et al.\xspace}
\newcommand{\wrt}{w.r.t.\xspace}

% text for operators
\newcommand{\operatortext}[1]{\mbox{\emph{#1}}\xspace}

\newcommand{\projectiontext}{\operatortext{projection}}
\newcommand{\selectiontext}{\operatortext{selection}}
\newcommand{\antijointext}{\operatortext{antijoin}}
\newcommand{\jointext}{\operatortext{natural join}}
\newcommand{\leftouterjointext}{\operatortext{left outer join}}
\newcommand{\cartesianproducttext}{\operatortext{Cartesian product}}
\newcommand{\uniontext}{\operatortext{union}}
\newcommand{\baguniontext}{\operatortext{bag union}}
\newcommand{\intersectiontext}{\operatortext{intersection}}
\newcommand{\minustext}{\operatortext{minus}}
\newcommand{\renametext}{\operatortext{rename}}

\newcommand{\alldifferenttext}{\operatortext{all-different}}
\newcommand{\duplicateeliminationtext}{\operatortext{duplicate-elimination}}
\newcommand{\sorttext}{\operatortext{sorting}}
\newcommand{\groupingtext}{\operatortext{grouping}}
\newcommand{\toptext}{\operatortext{top}}

\newcommand{\expandouttext}{\operatortext{expand-out}}
\newcommand{\expandintext}{\operatortext{expand-in}}
\newcommand{\expandbothtext}{\operatortext{expand-both}}

\newcommand{\getverticestext}{\operatortext{get-vertices}}
\newcommand{\getedgestext}{\operatortext{get-edges}}

\newcommand{\yes}{$\mdlgblkcircle$\xspace}
\newcommand{\no}{$\mdlgwhtcircle$\xspace}

\newcommand{\append}{\,\|\,}
\newcommand{\appendtext}{\operatortext{append}}
\newcommand{\remove}{-}
\newcommand{\removetext}{\operatortext{remove}}

\newcommand{\breakable}[2][c]{%
	\begin{tabular}[#1]{@{}l@{}}#2\end{tabular}}

\newcommand{\vertexlabels}{L_v}
\newcommand{\edgelabels}{L_e}

\newcommand{\verticestoedges}{\mathsf{src\_trg}}
\newcommand{\propertyfunction}[2]{\mathrm{#1}(#2)}

\newcommand{\vertexproperties}{P_v}
\newcommand{\edgeproperties}{P_e}

\newcommand{\vertexproperty}[1]{p_{#1}}
\newcommand{\edgeproperty}[1]{p_{#1}}

\newcommand{\vertexlabelfunction}{l_v}
\newcommand{\edgelabelfunction}{l_e}

\newcommand{\assign}{\rightarrow}

\newcommand{\attr}[1]{\mathrm{attr}(#1)}
\newcommand{\dom}[1]{\mathrm{dom}(#1)}
\newcommand{\schema}[1]{\mathrm{sch}(\mathsf{#1})}

%\newcommand{\relnull}{\bot}
%\newcommand{\relnull}{-}
\newcommand{\relnull}{\varepsilon}
\newcommand{\bigunion}{\bigcup}
\newcommand{\op}[2]{\mathrm{#1}\left(#2\right)}

\newcommand{\relnull}{\mathsf{NULL}}

\newcommand{\assign}{\rightarrow}

\newcommand{\lxor}{\oplus}

\newcommand{\asc}{\uparrow}
\newcommand{\desc}{\downarrow}

\newcommand{\tuple}[1]{\langle #1 \rangle}
\newcommand{\concatenation}{\Vert}

\newcommand{\literal}[1]{\mathsf{#1}}
\newcommand{\atom}[1]{\mathsf{#1}}

\newcommand{\var}[1]{\mathtt{#1}}
\newcommand{\edgevariable}[2]{\var{#1}
	\ifstrempty{#2}{}{\colon{\atom{#2}}}}
\newcommand{\nodevariable}[2]{(\var{#1}
	\ifstrempty{#2}{}{\colon{\atom{#2}}})}

% see http://tug.ctan.org/info/symbols/comprehensive/symbols-a4.pdf

%%%%%%%%%%%%%%%%%%%%% operator symbols %%%%%%%%%%%%%%%%%%%%%

\iftoggle{textualoperators}{
	% nullary operators
	\newcommand{\getverticesop}{\bigcirc}
	\newcommand{\getedgesopdirected}{\Uparrow}
	\newcommand{\getedgesopundirected}{\Updownarrow}
	
	% unary operators
	\newcommand{\expandbothop}{\updownarrow}
	\newcommand{\expandoutop}{\uparrow}
	\newcommand{\expandinop}{\downarrow}
	\newcommand{\transitiveclosurebothop}{\oplus \updownarrow}
	\newcommand{\transitiveclosureoutop}{\oplus \updownarrow}
	\newcommand{\transitiveclosureinop}{\oplus \updownarrow}
	\newcommand{\alldifferentop}{\not\equiv}
	\newcommand{\duplicateeliminationop}{\delta}
	\newcommand{\sortop}{\tau}
	\newcommand{\projectionop}{\pi}
	\newcommand{\selectionop}{\sigma}
	\newcommand{\renameop}{\rho}
	\newcommand{\groupingop}{\gamma}
	\newcommand{\topop}{\lambda}
	\newcommand{\unwindop}{\omega}
	
	% binary operators
	\newcommand{\joinop}{\bowtie}
	\def\ojoin{\setbox0=\hbox{$\bowtie$}%
		\rule[-.02ex]{.25em}{.4pt}\llap{\rule[\ht0]{.25em}{.4pt}}}
	\def\leftouterjoin{\mathbin{\ojoin\mkern-5.8mu\bowtie}}
	\def\rightouterjoin{\mathbin{\bowtie\mkern-5.8mu\ojoin}}
	\def\fullouterjoin{\mathbin{\ojoin\mkern-5.8mu\bowtie\mkern-5.8mu\ojoin}}
	
	\newcommand{\leftouterjoinop}{\leftouterjoin}
	\newcommand{\antijoinop}{\, \triangleright \,}
	\newcommand{\unionop}{\cup}
	\newcommand{\bagunionop}{\uplus}
	\newcommand{\minusop}{\setminus}
	\newcommand{\intersectionop}{\cap}
	\newcommand{\cartesianproductop}{\times}
}{
	\newcommand{\op}[1]{\textsc{#1}}
	
	% nullary operators
	\newcommand{\getverticesop}{\relalgop{GetVertices}}
	\newcommand{\getedgesopdirected}{\relalgop{GetEdges}}
	\newcommand{\getedgesopundirected}{\relalgop{GetEdgesUndirected}}
	
	% unary operators
	\newcommand{\expandbothop}{\relalgop{ExpandBoth}}
	\newcommand{\expandoutop}{\relalgop{ExpandOut}}
	\newcommand{\expandinop}{\relalgop{ExpandIn}}
	\newcommand{\transitiveclosurebothop}{\relalgop{TransitiveClosureBoth}}
	\newcommand{\transitiveclosureoutop}{\relalgop{TransitiveClosureOut}}
	\newcommand{\transitiveclosureinop}{\relalgop{TransitiveClosureIn}}
	\newcommand{\alldifferentop}{\relalgop{AllDifferent}}
	\newcommand{\duplicateeliminationop}{\relalgop{DuplicateElimination}}
	\newcommand{\sortop}{\relalgop{Sort}}
	\newcommand{\projectionop}{\relalgop{Projection}}
	\newcommand{\selectionop}{\relalgop{Selection}}
	\newcommand{\renameop}{\relalgop{Rename}}
	\newcommand{\groupingop}{\relalgop{Grouping}}
	\newcommand{\topop}{\relalgop{Top}}
	\newcommand{\unwindop}{\relalgop{Unwind}}
	
	% binary operators
	\newcommand{\joinop}{\relalgop{Join}}
	\newcommand{\leftouterjoinop}{\relalgop{LeftOuterJoin}}
	\newcommand{\antijoinop}{\relalgop{AntiJoin}}
	\newcommand{\unionop}{\relalgop{Union}}
	\newcommand{\bagunionop}{\relalgop{BagUnion}}
	\newcommand{\minusop}{\relalgop{Minus}}
	\newcommand{\intersectionop}{\relalgop{Intersection}}
	\newcommand{\cartesianproductop}{\relalgop{CartesianProduct}}
	
	% ternary operators
	%\newcommand{}{}
}

%%%%%%%%%%%%%%%%%%%%% operator definitions %%%%%%%%%%%%%%%%%%%%%

%%%%%%%%%% nullary operators %%%%%%%%%%

\newcommand{\getvertices}[2]{\getverticesop_{\nodevariable{#1}{#2}}}

\newcommand{\getedgesdirected}[6]{\getedgesopdirected_{\nodevariable{#1}{#2}}^{\nodevariable{#3}{#4}} \left[ \edgevariable{#5}{#6} \right]}
\newcommand{\getedgesundirected}[6]{\getedgesopundirected_{\nodevariable{#1}{#2}}^{\nodevariable{#3}{#4}} \left[ \edgevariable{#5}{#6} \right]}

%%%%%%%%%% unary operators %%%%%%%%%%
\newcommand{\nagivationbody}[3]{~_{\nodevariable{#1}{}}^{\nodevariable{#2}{#3}}}

\newcommand{\kleenestar}{\ast}

% expand operators
\newcommand{\expandedgevariable}[4]{
	\left[
	  % #3: minHops, cannot be empty
	  % #4: maxHops, if empty, default to infinity
	  \edgevariable{#1}{#2}
	  \ifstrequal{#3}{1} % minHops = 1
	  {
	  	\ifstrequal{#4}{1}
	  	{} % minHops = 1 and maxHops = 1 -> write nothing
	    {\kleenestar \atom{#3} \ldots \atom{#4}} % minHops = 1 and maxHops != 1
	  } % minHops != 1
      {\kleenestar \atom{#3} \ldots \atom{#4}}
	\right]}

\newcommand{\expandboth}[7]{\expandbothop \nagivationbody{#1}{#2}{#3} \expandedgevariable{#4}{#5}{#6}{#7} }
\newcommand{\expandout}[7]{\expandoutop \nagivationbody{#1}{#2}{#3} \expandedgevariable{#4}{#5}{#6}{#7} }
\newcommand{\expandin}[7]{\expandinop \nagivationbody{#1}{#2}{#3} \expandedgevariable{#4}{#5}{#6}{#7} }

% transitive closure operators
\newcommand{\transitiveclosureboth}[7]{\transitiveclosurebothop \nagivationbody{#1}{#2}{#3} \expandedgevariable{#4}{#5}{#6}{#7} }

\newcommand{\transitiveclosureout}[7]{\transitiveclosureoutop \nagivationbody{#1}{#2}{#3} \expandedgevariable{#4}{#5}{#6}{#7} }

\newcommand{\transitiveclosurein}[7]{\transitiveclosureinop \nagivationbody{#1}{#2}{#3} \expandedgevariable{#4}{#5}{#6}{#7} }

\newcommand{\alldifferent}[1]{\alldifferentop_{\atom{#1}}}
\newcommand{\duplicateelimination}{\duplicateeliminationop}
\newcommand{\sort}[1]{\sortop_{#1}}
\newcommand{\projection}[1]{\projectionop_\atom{#1}}
\newcommand{\selection}[1]{\selectionop_\atom{#1}}
\newcommand{\rename}[1]{\renameop_\atom{#1}}
\newcommand{\grouping}[1]{\groupingop_\atom{#1}}

% top/skip/limit operators
\newcommand{\topp}[2]{\topop_{#1}^{#2}}
\newcommand{\skipp}[1]{\topop^{#1}}
\newcommand{\limit}[1]{\topop_{#1}}

% see A Formal Presentation of MongoDB (Extended Version)
% by Elena Botoeva, Diego Calvanese, Benjamin Cogrel, Martin Rezk, Guohui Xiao
% https://arxiv.org/abs/1603.09291
\newcommand{\unwind}[2]{\unwindop_\atom{#1}^\atom{#2}}

%%%%%%%%%% binary operators %%%%%%%%%%

\newcommand{\join}{\joinop}
\newcommand{\myleftouterjoin}{\leftouterjoinop}
\newcommand{\antijoin}{\antijoinop}
\newcommand{\union}{\unionop}
\newcommand{\bagunion}{\bagunionop}
\newcommand{\minus}{\minusop}
\newcommand{\intersection}{\intersectionop}
\newcommand{\cartesianproduct}{\cartesianproduct}

% ternary operators


\graphicspath{{./figures/}}

\usepackage{amsmath}
\usepackage{stix}
\usepackage{mathtools}
\usepackage{tikz}
\usepackage{tikz-qtree}
\usetikzlibrary{shapes.gates.logic.US,trees,positioning,arrows}
\tikzset{
	level distance=1.5cm,
	sibling distance=1cm,
	edge from parent path={(\tikzparentnode.south) -- (\tikzchildnode.north)},
	every node/.style={draw},
	every tree node/.style={align=center,anchor=north}
}

\makeatletter
%\renewcommand{\paragraph}{%
%	\@startsection{paragraph}{4}%
%	{\z@}{0.25ex \@plus 1ex \@minus .2ex}{-1em}%
%	{\normalfont\normalsize\bfseries}%
%}
%\renewcommand{\paragraph}{%
%	\@startsection{paragraph}{4}%
%	{\z@}{0ex \@plus 0ex \@minus 0ex}{-1em}%
%	{\normalfont\normalsize\bfseries}%
%}
\renewcommand{\paragraph}[1]{\textbf{#1}}

\makeatother

\sloppy


\author{Gábor Szárnyas\footnote{Budapest University of Technology and Economics, Fault Tolerant Systems Research Group, MTA-BME Lendület Research Group on Cyber-Physical Systems, szarnyas@mit.bme.hu}\,\,\,
József Marton\footnote{Budapest University of Technology and Economics, Database Laboratory, marton@db.bme.hu}}

\newcommand{\shorttitle}{Formalizing \opencypher Graph Queries in Relational Algebra}
\title{Formalizing \opencypher Graph Queries\\ in Relational Algebra}

\begin{document}
\maketitle

%hint by http://tex.stackexchange.com/a/30229/9075 and http://tex.stackexchange.com/a/247652/9075
\thispagestyle{lnifirstpage}
\pagestyle{lni}

%Auf Anzahl der AutorInnen setzen, damit die weitere Nummerierung der Fußnoten passt
%Dieser Befehl \verb|\setcounter{footnote}{2}| legt in dem Fall fest, dass 2 Fußnotennummern bereits für die AutorInnen verbraucht wurden, damit die darauf folgenden Fußnoten mit der richtigen Nummerierung (ab 3) fortfahren. Dieser Wert muss an die jeweilige Zahl an AutorInnen bzw. bereits verbrauchte Fußnoten angepasst werden, sofern im weiteren Verlauf Fußnoten verwendet werden.
\setcounter{footnote}{2}

\begin{abstract}
The last decade brought considerable improvements in non-relational storage and query technologies, known collectively as NoSQL systems.
Most of these systems were designed to suit the needs of Big Data and real-time web applications, hence they use specialized data models: key-value, document, wide columns, graph databases, etc.
While they offer high performance for their specific use cases, a major shortcoming of these systems is the lack of standardization. Consequently, migrating datasets or applications between technologies often requires a large amount of manual work or ad-hoc solutions, thus subjecting the users to the possibility of vendor lock-in.

To avoid this threat, NoSQL vendors are working on supporting existing standard languages (\eg SQL) and introducing standardized languages. In this paper, we present formal definitions for openCypher, a high-level declarative graph query language with an on-going standardization effort.

\end{abstract}

%\begin{keywords}
%graph databases, openCypher
%\end{keywords}

\chapter{Introduction}

The introduction -- much to your surprise -- comes here.


\section{Preliminaries}
\label{sec:preliminaries}

This section defines the mathematical concepts used in the paper. Our notation closely follows~\cite{DBLP:conf/edbt/HolschG16} and is similar to~\cite{DBLP:books/igi/Sakr11/RodriguezN11}\footnote{The formalism presented in~\cite{DBLP:books/igi/Sakr11/RodriguezN11} lacks the notion of \emph{vertex labels}.}.

\subsection{Property Graph Data Model}

A \emph{property graph} is defined as $G = (V, E, \verticestoedges, \vertexlabels, \edgelabels, \vertexlabelfunction, \edgelabelfunction, \vertexproperties, \edgeproperties)$, where $V$ is a set of vertices, $E$ is a set of directed edges, $\verticestoedges: E \assign V \cartesianproductop V$ assigns the source and target vertices to edges. The graph is labelled (or typed):
\begin{itemize}
	\item $\vertexlabels$ is a set of vertex labels, $\vertexlabelfunction: V \assign 2^{\vertexlabels}$ assigns \emph{a set of labels} to each vertex.
	\item $\edgelabels$ is a set of edge labels, $\edgelabelfunction: E \assign \edgelabels$ assigns \emph{a single label} to each edge.
\end{itemize}

Furthermore, graph $G$ has properties (\emph{attributed graph}). Let $D$ be a set of atomic domains.
\begin{itemize}
	\item $\vertexproperties$ is a set of vertex properties. A vertex property $p_i \in \vertexproperties$ is a function $p_i: V \assign D_i \unionop \{ \relnull \}$, which assigns a property value from a domain $D_i \in D$ to a vertex $v \in V$, if $v$ has property $p_i$, otherwise $p_i(v)$ returns $\relnull$.
	\item $\edgeproperties$ is a set of edge properties. An edge property $p_j \in \edgeproperties$ is a function $p_j: E \assign D_j \unionop \{ \relnull \}$, which assigns a property value from a domain $D_j \in D$ to an edge $e \in E$, if $e$ has property $p_j$, otherwise $p_j(e)$ returns $\relnull$.
\end{itemize}

\begin{figure}
	\centering
	\includegraphics[width=6cm]{figures/movie-graph}
	\caption{Example movie graph.}
	\label{fig:running-example-property-graph}
\end{figure}

\paragraph{Running example.} \autoref{fig:running-example-property-graph} presents an example inspired by the Movie Database dataset\footnote{\url{https://neo4j.com/developer/movie-database/}}. The graph can be represented formally as:

\begin{minipage}{\textwidth}
	$V=\{1, 2, 3, 4, 5\}; E=\{11,12,13,14,15\};$
	
	$\verticestoedges(11) = \tuple{1, 2}; \verticestoedges(12) = \tuple{3, 2}; \ldots$

	$\vertexlabels = \{\atom{Actor}, \atom{Director}, \atom{Movie}\};$

	$\edgelabels = \{\atom{ACTS\_IN}, \atom{DIRECTED}\};$

	$\vertexlabelfunction(1) = \{\atom{Actor}, \atom{Director}\}; \vertexlabelfunction(2) = \{\atom{Movie}\}; \ldots;$

	$\edgelabelfunction(11) = \atom{ACTS\_IN}; \edgelabelfunction(12) = \atom{DIRECTED}; \ldots;$

	$\vertexproperties = \{\atom{name}, \atom{title}, \atom{release}\}; \edgeproperties = \{\};$

	$\propertyfunction{name}{1} = \atom{'Clint~Eastwood'}; \propertyfunction{name}{2} = \relnull; \ldots$

	$\propertyfunction{title}{1} = \relnull; \propertyfunction{title}{2} = \atom{'The~Good,~the~Bad~and~the~Ugly'}; \ldots$
	
	$\propertyfunction{release}{1} = \relnull; \propertyfunction{release}{2} = \atom{1966}; \ldots$
\end{minipage}

In the context of this paper, we define a \emph{relation} as a bag (\emph{multiset}) of tuples: a tuple can occur more than once in the relation~\cite{DBLP:books/daglib/0020812}.
Given a property graph $G$, relation $r$ is a \emph{graph relation} if the following holds:
$$\forall A \in \attr{r}: \dom{A} \subseteq V \union E \union D,$$
where $\attr{r}$ is the set of attributes of $r$, $\dom{A}$ is the domain of attribute $A$. The schema of $r$, $\schema{r}$ is a list containing the attribute names. For schema transformations, the \appendtext operator is denoted by $\append$, the \removetext operator is denoted by $\remove$.

\section{Operators of Relational Algebra}

For well-known relational algebra operators (\eg selection, projection, join) and common extensions (\eg aggregation, left outer join), we only give a brief summary. A more detailed discussion is available in database textbooks, \eg~\cite{DBLP:books/daglib/0020812, DBLP:books/daglib/0006733}.

We also adapted graph-specific operators from~\cite{DBLP:conf/edbt/HolschG16}\footnote{The \textsc{GetNodes} operator introduced in~\cite{DBLP:conf/edbt/HolschG16} and did not support labels. We extended it by allowing the specification of vertex labels and renamed it to \getverticestext to be consistent with the rest of the definitions. We also extended the \textsc{ExpandIn} and \textsc{ExpandOut} operators to allow it to return a set of edges, and introduced the \expandbothtext operator to allow navigation to both directions.} and propose new operators.

\subsection{Nullary Operators}
\label{sec:nullary-operators}

The \getverticestext nullary operator $\getvertices{v}{t_1 \land \ldots \land t_n}$ returns a graph relation of a single attribute $v$ that contains the ID of all vertices that have \emph{all} of labels $t_1, \ldots, t_n$.

\subsection{Unary Operators}
\label{sec:unary-operators}

The \projectiontext operator $\projectionop$ keeps a specific set of attributes in the relation: $ t = \projection{A_1, \ldots, A_n} \left(r\right).$ Note that the tuples are not deduplicated by default, \ie the results will have the same number of tuples as the input relation $r$. The projection operator can also rename the attributes, \eg $\projection{v1 \assign v2} \left(r\right)$ renames $\atom{v1}$ to $\atom{v2}$.

The \selectiontext operator $\selectionop$ filters the incoming relation according to some criteria. Formally,
$ t = \selection{\theta} \left(r\right), $
where predicate $\theta$ is a propositional formula. The operator selects all tuples in $r$ for which $\theta$ holds.

The \duplicateeliminationtext operator $\duplicateeliminationop$ eliminates duplicate tuples in a bag.

The \groupingtext operator $\groupingop$ groups tuples according to their value in one or more attributes and aggregates the remaining attributes. %Aggregated values (scalars and inline collections) makes \rga not closed under \groupingtext.

The \sorttext operator $\sortop$ transforms a bag relation of tuples to a list of tuples by ordering them. The ordering is defined by specified attributes of the tuples with an ordering direction (ascending $\asc$/descending $\desc$) for each attribute, \eg $\sort{\asc \atom{v1}, \desc \atom{v2}} (r)$.

The \toptext operator $\topp{l}{s}$ (adapted from~\cite{DBLP:conf/sigmod/LiCIS05}) takes a list as its input, skips the top $s$ tuples and returns the next $l$ tuples.\footnote{SQL implementations offer the \texttt{OFFSET} and the \texttt{LIMIT}/\texttt{TOP} keywords.}

The \expandbothtext operator $\expandboth{E}{l_1 \lor \ldots \lor l_k \ast min \ldots max}{v}{w}{t_1 \land \ldots \land t_n}(r)$ adds (1)~a new attribute $w$ to $r$ containing the IDs of vertices having \emph{all} labels $t_1, \ldots, t_n$ that can be reached from vertices of attribute $v$ by traversing edges having \emph{any} labels $l_1, \ldots, l_n$, and (2)~a new attribute $E$ for the edges of the path from $v$ to $w$. The operator may use at least $\atom{min}$ and at most $\atom{max}$ hops, both defaulting to $1$ if omitted. % With the default setting, \ie $\atom{min}=\atom{max}=1$, a single edge variable $e$ can be used instead of edge list $E$.
The \expandintext operator~$\expandinop$ and \expandouttext operator~$\expandoutop$ only consider directed paths from $w$ to $v$ and from $v$ to $w$, respectively.

The \unwindtext operator $\unwind{xs}{x}$ unfolds a list $\atom{xs}$ to a variable $\atom{x}$ by introducing additional rows, each containing a single element of the list.

The \alldifferenttext operator $\alldifferent{E_1, E_2, E_3, \ldots}{(r)}$ filters $r$ to keep tuples where the variables in $\bigcup_{i} E_{i}$ are pairwise different.\footnote{Should e.g. $E_2$ be a set of the single variable $e_2$, the variable name can be used as a shorthand instead, so $\alldifferent{E_1, e_2, E_3, \ldots}{(r)} ~ \equiv ~ \alldifferent{E_1, \{e_2\}, E_3, \ldots}{(r)}$}
It can be expressed as a \selectiontext:
$$\alldifferent{E_1, E_2, E_3, \ldots}{(r)} = \selection{ \bigwedge\limits_{e_1, e_2 \,\in\, \bigcup\limits_{i} {E_i} ~ \wedge ~ { e_1 \,\neq\, e_2 } } { r.e_1 \,\neq\, r.e_2 } }{(r)}$$

We use the \alldifferenttext operator to guarantee the uniqueness of edges (see the remark on \emph{uniqueness of edges} in \autoref{sec:opencypher}).

\subsection{Binary Operators}
\label{sec:binary-operators}

The $\unionop$ operator produces the set union of two relations, while the $\bagunionop$ operator produces the \emph{bag union} of two operators, \eg $\{\tuple{1, 2}, \tuple{1, 2}, \tuple{3, 4}\} \bagunionop \{\tuple{1, 2}\} = \{\tuple{1, 2}, \tuple{1, 2}, \tuple{1, 2}, \tuple{3, 4}\}$. For both the \uniontext and \baguniontext operators, the schema of the operands must have the same number of attributes. Some authors also require that they share a common schema, \ie have the same set of attributes~\cite{DBLP:books/daglib/0020812}.

The $\cartesianproductop$ operator produces the \cartesianproducttext:

$$ t = r \cartesianproductop s.$$

The result of the \jointext operator $\joinop$ is determined by creating the Cartesian product of the relations, then filtering those tuples which are equal on the attributes that share a common name. The combined tuples are projected: from the attributes present in both of the two input relations, we only keep the ones in $r$ and drop the ones in $s$. Thus, the join operator is defined as
$$r \join s = \pi_{R \union S} \left(\selection{r.A_1 = s.A_1\,\land\,\ldots\,\land\,r.A_n = s.A_n)} \left(r \times s\right) \right),$$
where $ \{ A_1, \ldots, A_n \} $ is the set of attributes that occur both in $R$ and $S$, \ie $ R \intersection S = \{ A_1, \ldots, A_n \} $. Note that if the set of common attributes is empty, the \jointext operator is equivalent to the Cartesian product of the relations.
The join operator is both commutative and associative: $r \join s = s \join r$ and $(r \join s) \join t = r \join (s \join t)$, respectively.

The \antijointext operator $\antijoinop$ (also known as \emph{left anti semijoin}) collects the tuples from the left relation $r$ which have no matching pair in the right relation $s$:
$$ t = r \antijoin s = r \setminus \pi_{R} \left(r \join s\right), $$
where $\pi_{R}$ denotes a projection operator, which only keeps the attributes of the schema over relation $r$. The antijoin operator is not commutative and not associative.

The \leftouterjointext $\myleftouterjoin$ pads tuples from the left relation that did not match any from the right relation with $\relnull$ values and adds them to the result of the \jointext~\cite{DBLP:books/daglib/0015084}:
$$ t = r \myleftouterjoin s = (r \join s) \union (r \antijoin s) \cartesianproductop \{\relnull, \ldots, \relnull\}, $$

where the constant relation $\{\relnull, \ldots, \relnull\}$ is on the schema $S \setminus R$.

\subsection{Property Access}

Assuming that $x$ is an attribute of a graph relation, we use the notation $x.a$ in (1)~attribute lists for projections and (2)~selection conditions to express the access to the corresponding value of property $a$ in the property graph~\cite{DBLP:conf/edbt/HolschG16}.

\subsection{Summary of Operators}

\autoref{table:collections} provides an overview of the operators of \rga.

\newcommand{\propheader}{\multirow{2}{*}{\bf prop.}}
\newcommand{\rgaheader}{\multirow{2}{*}{\breakable{\bf RGA}}}

\setlength\tabcolsep{3.6pt}
\begin{table}[htb]
	\centering
	\begin{tabular}{||c||c|c|c||c|c|c||c||c||}
		\hline
		\multirow{2}{*}{\bf ops} &             \multirow{2}{*}{\bf operator}             &         \multirow{2}{*}{\bf name}         & \propheader & \multicolumn{3}{c||}{\bf output for} &             \multirow{2}{*}{\bf schema}              \\ \cline{5-7}
		&                                                       &                                           &             & \bf set & \bf bag &     \bf list     &  \\ \hline\hline
		\multirow{1}{*}{\bf 0}   &                  $\getvertices{v}{}$                  &             \getverticestext              &     set     &   set   &   set   &       set        &                  $\tuple{\atom{v}}$                  \\ \hline\hline %\cline{2-8}
%		&              $\getedges{v}{}{w}{}{e}{}$               &               \getedgestext               &     $-$     &   $-$   &   $-$   &       $-$        &               $\tuple{\atom{v, e, w}}$               \\ \hline\hline
		\multirow{8}{*}{\bf 1}   &         $\projection{v_1, v_2, \ldots} (r)$         &              \projectiontext              &      i      &   bag   &   bag   &       list       &         $\tuple{\atom{v_1, v_2, \ldots}}$          \\ \cline{2-8}
		&              $\selection{condition} (r)$              &              \selectiontext               &      i      &   set   &   bag   &       list       &                     $\schema{r}$                     \\ \cline{2-8}
		&            $\expandboth{v}{w}{}{e}{}{min}{max} (r)$             &              \expandbothtext              &     $-$     &   set   &   bag   &       list       &       $\schema{r} \append \tuple{\atom{e, w}}$       \\ \cline{2-8}
		&            $\alldifferent{variables} (r)$             &             \alldifferenttext             &      i      &   set   &   bag   &       list       &                     $\schema{r}$                     \\ \cline{2-8}
		&             $\duplicateeliminationop (r)$             &         \duplicateeliminationtext         &      i      &   set   &   set   &       list       &                     $\schema{r}$                     \\ \cline{2-8}
		&                     $\sort{\desc \atom{v_1}, \asc \atom{v_2}, \ldots} (r)$                     &                 \sorttext                 &      i      &  list   &  list   &       list       &                     $\schema{r}$                     \\ \cline{2-8}
		&          $\grouping{v_1, v_2, \ldots} (r)$          &               \groupingtext               &      i      &   set   &   set   &       set        &         $\tuple{\atom{v_1, v_2, \ldots}}$          \\ \cline{2-8}
		&                     $\topop (r)$                      &                 \toptext                  &     $-$     &  list   &  list   &       list       &                     $\schema{r}$                     \\ \hline\hline
		\multirow{5}{*}{\bf 2}   & $r \unionop s$, $r \minusop s$ & \uniontext, \minustext &     $-$     &   set   &   set   &       set        &                     $\schema{r}$                     \\ \cline{2-8}
		&                    $r \bagunion s$                    &               \baguniontext               &    c, a     &   bag   &   bag   &       bag        &                     $\schema{r}$                     \\ \cline{2-8}
		&               $r \cartesianproductop s$               &           \cartesianproducttext           &    c, a     &   set   &   bag   &       bag        &           $\schema{r} \append \schema{s}$            \\ \cline{2-8}
		&                    $r \joinop s$                      &                 \jointext                 &    c, a     &   set   &   bag   &       bag        & $\schema{r} \append (\schema{s} \minus \schema{r}) $ \\ \cline{2-8}
		&                $r \myleftouterjoin s$                 &            \leftouterjointext             &     $-$     &   set   &   bag   &       bag        & $\schema{r} \append (\schema{s} \minus \schema{r}) $ \\ \cline{2-8}
		&                   $r \antijoinop s$                   &               \antijointext               &    c, a     &   set   &   bag   &       bag        &                     $\schema{r}$                     \\ \cline{1-8}
	\end{tabular}
	\caption{Properties of relational graph algebra operators. A unary operator $\alpha$ is idempotent~(i), iff $\alpha(x) = \alpha(\alpha(x))$ for all inputs. A binary operator $\beta$ is commutative~(c), iff $x~\beta~y = y~\beta~x$ and associative~(a), iff $(x~\beta~y)~\beta~z = x~\beta~(y~\beta~z)$.}
	\label{table:collections}
\end{table}


\section{The openCypher Query Language}
\label{sec:opencypher}

\paragraph{Language.} As the primary query language of Neo4j~\cite{Neo4j}, Cypher~\cite{Cypher} was designed to read easily. It allows users to specify the graph pattern by a syntax resembling an actual graph. The goal of the \opencypher project~\cite{openCypher} is to provide a standardised specification of the Cypher language.
\autoref{lst:example} shows an \opencypher query, which returns all people who (1)~are both actors and directors and (2)~have acted in a movie together with $\atom{Clint\ Eastwood}$.\\

\begin{minipage}{\linewidth}
\begin{lstlisting}[label=lst:example, caption={Get people who are both actors and directors and acted in a movie with Clint Eastwood.}]
MATCH (a1)-[:ACTS_IN]->(:Movie)<-[:ACTS_IN]-(a2:Actor:Director)
WHERE a1.name = 'Clint Eastwood'
RETURN a2
\end{lstlisting}
\end{minipage}

The query returns with a bag of vertices that have both the labels $\atom{Actor}$ and $\atom{Director}$ and share a common $\atom{Movie}$ neighbor through $\atom{ACTS\_IN}$ edges. Cypher guarantees that these edges are only traversed once, so the vertex of $\atom{Clint\ Eastwood}$ is not returned (see the section on the uniqueness of edges).

\paragraph{Implementation.} While Neo4j uses a parsing expression grammar (PEG)~\cite{DBLP:conf/popl/Ford04} for specifying the grammar rules of Cypher, openCypher aims to achieve an implementation-agnostic specification by only providing a context-free grammar. The parser can be implemented using any capable parser technology, \eg \antlr~\cite{Parr:2013:DAR:2501720} or Xtext~\cite{DBLP:conf/oopsla/EysholdtB10}. %It also possible to generate a grammar following the ISO 14977 Extended Backus--Naur Form~\cite{ISO14977},

\paragraph{Legacy grammar rules.} It is not a goal of the openCypher project to fully cover the features of Neo4j's Cypher language: ``Not all grammar rules of the Cypher language will be standardised in their current form, meaning that they will not be part of openCypher as-is. Therefore, the openCypher grammar will not include some well-known Cypher constructs; these are called 'legacy'.''\footnote{\url{https://github.com/opencypher/openCypher/tree/master/grammar}} The \emph{legacy rules} include commands (\lstinline+CREATE INDEX+, \lstinline+CREATE UNIQUE CONSTRAINT+, etc.), pre-parser rules (\lstinline+EXPLAIN+, \lstinline+PROFILE+) and deprecated constructs (\lstinline+START+). A detailed description is provided in the openCypher specification. In our work, we focused on the \emph{standard core} of the language and ignored legacy rules.

% http://neo4j.com/docs/developer-manual/current/cypher/#cypherdoc-uniqueness
\paragraph{Uniqueness for edges.} In an \opencypher query, a \lstinline+MATCH+ clause defines a graph pattern. A query can be composed of multiple patterns spanning multiple \lstinline+MATCH+ clauses. For the matches of a pattern within a single \lstinline+MATCH+ clause, edges are required to be unique. %even for disconnected graph patterns.
However, matches for multiple \lstinline+MATCH+ clauses can share edges. This uniqueness criterium can be expressed in a compact way with the \alldifferenttext operator introduced in \autoref{sec:unary-operators}. For vertices, this restriction does not apply.

\paragraph{Aggregation.} It indeed makes sense to calculate aggregation over graph pattern matches, though, its result will not necessarily be pattern match with vertices and edges. Based on some \emph{grouping criteria}, matches are put into categories, and values for the grouping criteria as well as grouping functions over the groups, the aggregations are evaluated in a single tuple for each and every category. For example, \lstinline+count+, \lstinline+avg+, \lstinline+sum+, \lstinline+max+, \lstinline+min+, \lstinline+stdDev+, \lstinline+stdDevP+, \lstinline+collect+. The \lstinline+collect+ function is an exception as it does not return a single scalar value but returns a collection (list).

In the SQL query language, grouping criteria is explicitly given by using the \lstinline+GROUP BY+ clause. In \opencypher, however, this is done implicitly in the \lstinline+RETURN+ as well as in \lstinline+WITH+ clauses: vertices, edges and their properties that appear outside the grouping functions become the \emph{grouping criteria}.\footnote{This approach is also used by some SQL code assistant IDEs generating the \lstinline+GROUP BY+ clause for a query.}

\paragraph{Subqueries.} One can compose an \opencypher query of multiple subqueries. Subqueries, written subsequently, mostly begin by a \lstinline+MATCH+ clause and end at (including) a \lstinline+RETURN+ or \lstinline+WITH+ clause, the latter having an optional \lstinline+WHERE+ clause to follow. The \lstinline+WITH+ and \lstinline+RETURN+ clauses determine the resulting schema of the subquery by specifiying the vertices, edges, attributes and aggregates of the result. When \lstinline+WITH+ has the optional \lstinline+WHERE+ clause, it applies an other filter on the subquery result.\footnote{This is much like the \lstinline+HAVING+ construct of the SQL language with the major difference that it is also allowed in \opencypher in case no aggregation has been done.} The last subquery must be ended by \lstinline+RETURN+, whereas all the previous ones must be ended by \lstinline+WITH+. If a query is composed by more than one subqueries, their results are joined together using \jointext or \leftouterjointext operators.


\section{Mapping \opencypher Queries to \RGA}
\label{sec:compilation}

In this section, we first give the mapping algorithm of \opencypher queries to \rga, then we give a more detailed listing of the compilation rules for the query language constructs in \cref{table:mapping}.
We follow the bottom-up approach to build the \rga tree based on the \opencypher query. The algorithm is as follows. Join operations always use all common variables to match the two inputs (see \jointext in \cref{sec:rga}).

\setlength\tabcolsep{3.6pt}
\begin{enumerate}
\label{alg:build-rga-tree}
	\item A single pattern is turned left-to-right to a \getverticestext for the first vertex and a chain of \expandintext, \expandouttext or \expandbothtext operators for inbound, outbound or undirected relationships, respectively.
	\item Patterns in the same \lstinline+MATCH+ clause are joined by \jointext.
	\item Append an \alldifferenttext operator for all edge variables that appear in the \lstinline+MATCH+ clause because of the non-repeating edges language rule.
	\item Process the \lstinline+WHERE+ clause. Note that according to the grammar, \lstinline+WHERE+ is bound to a \lstinline+MATCH+ clause.
	\item Several \lstinline+MATCH+ clauses are connected to a left deep tree of \jointext. If \lstinline+MATCH+ has the \lstinline+OPTIONAL+ modifier, \leftouterjointext is used instead of \jointext.
	\item If there is a positive or negative pattern deferred from \lstinline+WHERE+ processing,
		append it as a \jointext or \antijointext operator, respectively.
	\item Append \groupingtext, if \lstinline+RETURN+ or \lstinline+WITH+ clause has grouping functions inside
	\item Append \projectiontext operator based on the \lstinline+RETURN+ or \lstinline+WITH+ clause. This operator will also handle the renaming (i.e. \lstinline+AS+).
	\item Append \duplicateeliminationtext operator, if the \lstinline+RETURN+ or \lstinline+WITH+ clause has the \lstinline+DISTINCT+ modifier.
	\item Append a \selectiontext operator in case the \lstinline+WITH+ had the optional \lstinline+WHERE+ clause.
	\item If this is not the first subquery, join to the \rga tree using \jointext or \leftouterjointext.
	\item Assemble a \uniontext operation from the query parts\footnote{In this context, query parts refer to those parts of the query connected by the \lstinline+UNION+ \opencypher keyword.}. As the \uniontext operator is technically a binary operator, the \uniontext of more than two query parts are represented as a left deep tree of \lstinline+UNION+ operators.
\end{enumerate}

\setlength\extrarowheight{2.5pt}
\setlength\tabcolsep{3.6pt}
\begin{table}[htbp]
	\centering
	\begin{tabular}{|l|l|l|}
		\hline
		\multicolumn{2}{|l|}{ \bf Language construct } & \bf Relational algebra expression \\ \hline\hline

		%\hline
		\multicolumn{3}{|l|}{Vertex, edge and path patterns } \\ \cline{2-3}

		& \lstinline+()+ & $\getvertices{\_v}{}$ \\ \cline{2-3}

		& \lstinline+(:types)+ & $\getvertices{\_v}{types}$ \\ \cline{2-3}

		& \lstinline+(<v>v</v>:types)+ & $\getvertices{v}{types}$ \\ \cline{2-3}

		% expand operators
		& \lstinline+<v>p</v>-[<v>e</v>:<v>labels</v>]-(<v>w</v>:types...)+ & \multirow{2}{*}{$\expandboth{v}{w}{types}{e}{labels}{1}{1} (\atom{p})$} \\ \cline{2-2}

		& \lstinline+<v>p</v><-[<v>e</v>:<v>labels</v>]->(<v>w</v>:types...)+ & \\ \cline{2-3}

		& \lstinline+<v>p</v>-[<v>e</v>:<v>labels</v>]->(<v>w</v>:types...)+ & $\expandout{v}{w}{types}{e}{labels}{1}{1} (\atom{p})$ \\ \cline{2-3}

		& \lstinline+<v>p</v><-[<v>e</v>:<v>labels</v>]-(<v>w</v>:types...)+ & $\expandin{v}{w}{types}{e}{labels}{1}{1} (\atom{p})$ \\ \cline{2-3}

		& \lstinline+<v>p</v>-[<v>labels</v>*<v>min</v>..<v>max</v>]->(<v>w</v>:<v>t2</v>)+ & $\expandout{v}{w}{types}{E}{labels}{min}{max}(\atom{p})$ \\ \cline{2-3}

		\hline \multicolumn{3}{|l|}{Combining and filtering pattern matches } \\ \cline{2-3}

		& \lstinline+MATCH <v>p</v>+ & $\alldifferent{\atom{edges~of~p}} \left(\atom{p}\right)$ \\ \cline{2-3}

		& \lstinline+MATCH <v>p1</v>, <v>p2</v>+ &
		$\alldifferent{\atom{edges~of~p1~and~p2}} \left( \atom{p_1}~\join~\atom{p_2} \right)$ \\ \cline{2-3}

		& \breakable{
			\lstinline+MATCH <v>p1</v>+ \\
			\lstinline+MATCH <v>p2</v>+
		} &
		$\alldifferent{\atom{edges~of~p1}} \left(\atom{p_1}\right)~\join~\alldifferent{\atom{edges~of~p2}} \left(\atom{p_2}\right)$ \\ \cline{2-3}

		& \breakable{
			\lstinline+MATCH <v>p1</v>+ \\
			\lstinline+OPTIONAL MATCH <v>p2</v>+
		} & $\alldifferent{\atom{edges~of~p1}} \left(\atom{p_1}\right)~\myleftouterjoin~\alldifferent{\atom{edges~of~p2}} \left(\atom{p_2}\right)$ \\ \cline{2-3}

		& \breakable{
			\lstinline+MATCH <v>p</v>+ \\
			\lstinline+WHERE <v>condition</v>+
		} & \breakable{$\selection{\atom{condition}}{\left( r \right)}$, where $\atom{condition}$ may \tabularnewline specify patterns and arithmetic \tabularnewline constraints on existing variables} \\ \cline{2-3}

		\hline \multicolumn{3}{|l|}{Result and sub-result operations. Rules for \lstinline+RETURN+ also apply to \lstinline+WITH+.} \\ \cline{2-3}

		& \lstinline+RETURN <v>variables</v>+ & $\projection{\atom{variables}}{\left( r \right)}$ \\ \cline{2-3}

		& \lstinline+RETURN <v>v1</v> AS <v>alias1</v> ...+ & $\projection{\atom{v1} \assign \atom{alias1}, \ldots }\left( r \right)$ \\ \cline{2-3}

		& \lstinline+RETURN DISTINCT <v>variables</v>+ & $\duplicateelimination\left(\projection{\atom{variables}}{\left( r \right)}\right)$ \\ \cline{2-3}

		& \lstinline+RETURN <v>variables</v>, <v>aggregates</v>+ & $\grouping{\atom{variables}, \atom{aggregates}}{\left( r \right)}$ \\ \cline{2-3}

		\hline \multicolumn{3}{|l|}{List operations } \\ \cline{2-3}

		& \lstinline+ORDER BY <v>v1</v> [ASC|DESC] ...+ & $\sort{\asc/\desc \atom{v1}, \ldots}{\left( r \right)}$ \\ \cline{2-3}

		& \lstinline+LIMIT <v>l</v>+ & $\limit{l}(r)$ \\ \cline{2-3}

		& \lstinline+SKIP <v>s</v>+ & $\skipp{s}(r)$ \\ \cline{2-3}
		
		& \lstinline+SKIP <v>s</v> LIMIT <v>l</v>+ & $\topp{l}{s}(r)$ \\ \cline{2-3}
		
		\hline \multicolumn{3}{|l|}{Combining results } \\ \cline{2-3}
		
		& \lstinline+<v>query1</v> UNION <v>query2</v>+ & $r_1 \union r_2$ \\ \cline{2-3}

		& \lstinline+<v>query1</v> UNION ALL <v>query2</v>+ & $r_1 \bagunion r_2$ \\ \hline
	\end{tabular}
	\caption{Mapping from \opencypher constructs to relational algebra.}
	\label{table:mapping}
\end{table}

\paragraph{Example.} The example query in~\cref{lst:example} can be formalized as:
{\footnotesize
	\begin{align*}
	&\projection{a2} \Bigg(\selection{a1.name = 'C.\,E.'} \Big( \alldifferent{\_e1, \_e2} \expandin{a1}{a2}{Actor \land Director}{\_e1}{ACTS\_IN}{1}{1} \expandout{a1}{}{Movie}{\_e2}{ACTS\_IN}{1}{1} \left(\getvertices{a1}{Actor}\right) \Big) \Bigg)
	\end{align*}
}

Note that the $\alldifferentop$ guarantees the uniqueness constraint for the edges (\cref{sec:opencypher}), which prevents the query from returning the vertex $\atom{Clint~Eastwood}$.

\paragraph{Optimisations.} Queries with negative conditions for patterns can also be expressed using the \antijointext operator. For example, \lstinline+MATCH <v>p1</v> WHERE NOT <v>p2</v>+ can be formalized as
$$\alldifferent{\atom{edges~of~p1}} \left(\atom{p_1}\right) \antijoin \alldifferent{\atom{edges~of~p2}} \left(\atom{p_2}\right)$$

\paragraph{Limitations.} Our mapping does not completely cover the \opencypher language. As discussed in \cref{sec:opencypher}, some constructs are defined as legacy and thus were omitted. Also, we did not formalize expressions (\eg  conditions in selections), collections (arrays and maps), which are required for both path variables\footnote{\lstinline+MATCH p=(:Person)-[:FRIEND*1..2]->(:Person)+} and the \lstinline+UNWIND+ operator. The mapping does not cover parameters and data manipulation operations, \eg \lstinline+CREATE+, \lstinline+DELETE+, \lstinline+SET+ and \lstinline+MERGE+.


\section{Related Work}
\label{sec:related-work}

The TinkerPop framework~\cite{TinkerPop} aims to provide a standard data model for property graphs, along with Gremlin, a high-level graph-traversal language~\cite{Rodriguez:2015:GGT:2815072.2815073} and the Gremlin Structure API, a low-level programming interface.

Besides property graphs, graph queries can be formalized on different graph-like data models and even relational databases.

\paragraph{EMF.} The Eclipse Modeling Framework (EMF) is an object-oriented modelling framework widely used in model-driven engineering. 
Henshin~\cite{DBLP:conf/models/ArendtBJKT10} provides a visual language for defining patterns, while Epsilon~\cite{DBLP:conf/icmt/KolovosPP08} and \viatraquery~\cite{DBLP:conf/models/BergmannHRVBBO10} provide high-level declarative (textual) query languages, Epsilon Pattern Language and \vql.

\paragraph{RDF.} The Resource Description Framework (RDF)~\cite{RDF} aims to describe entities of the semantic web. RDF assumes sparse, ever-growing and incomplete data stored as triples that can be queried using the \sparql~\cite{SPARQL} graph pattern language.
%A formal definition of the \sparql language is given in~\cite{DBLP:journals/tods/PerezAG09}.

\lstset{language=}

\paragraph{SQL.} In general, relational databases offer limited support for graph queries: recursive queries are supported by \mbox{PostgreSQL} using the \lstinline+WITH RECURSIVE+ keyword and by the Oracle Database using the \lstinline+CONNECT BY+ keyword. Graph queries are supported in \saphana Graph
Scale-Out Extension prototype~\cite{DBLP:conf/btw/RudolfPBL13}, through a SQL-based language~\cite{DBLP:conf/gg/KrauseJDSKN16}.

\lstset{language=Cypher}


\section{Conclusion and Future Work}
\label{sec:conclusion}

In this paper, we presented a formal specification for a subset of the \opencypher query language. This provides the theoretical foundations to use \opencypher as a language for graph query engines. Using the proposed mapping, an \opencypher-compliant query engine could be built on any relational database engine to (1)~store property graphs as graph relations and to (2)~efficiently evaluate the extended operators of \rga.

As a future work, we will give formal specification of the operators for incremental query evaluation, which requires us to define \emph{maintenance operations} to keep their result in sync with the latest set of changes. Our long-term research objective is to design and prototype a \emph{distributed, incremental graph query engine}~\cite{DBLP:conf/models/SzarnyasIRHBV14} for the property graph data model.


\input{acknowledgements}

%\clearpage % try to keep the references down to a single page
\bibliography{bib}

%\input{appendix}

\end{document}
